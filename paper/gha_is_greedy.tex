% !TEX encoding = windows-1251
% !TEX program = pdflatex
\documentclass[11pt]{article}
\usepackage[cp1251]{inputenc}
%\usepackage[T2A]{fontenc}
%\usepackage[russian]{babel}
\usepackage{subcaption}
\usepackage{amssymb,amsmath}
\usepackage{multicol}
\pagestyle{empty}
\usepackage{graphicx}
\textheight=25cm
\textwidth=18cm
\oddsidemargin=-1cm
\topmargin=-2.5cm
\newtheorem{lem}{Lemma}
\DeclareMathOperator{\suff}{suff}
\DeclareMathOperator{\pref}{pref}
\DeclareMathOperator{\overlap}{overlap}

\begin{document}
	\begin{lem}
		Let $GHA(\mathcal{S})$ return a superstring corresponding to a permutation $\sigma = (s_{i_1}, s_{i_2}, \dots, s_{i_n})$. Then an algorithm $A$ that merges adjacent strings in $\sigma$ in the descending order of the lengths its overlaps is greedy.
	\end{lem}
	{\em Proof.} Let $c$ be a cycle through solution $D$ of $GHA(\mathcal{S})$ in the hierarchical graph and let us denote by $c_k$ a part of $c$ above level $k$ for any $k$. $c_k$ can be represented as disjoint union $\mathcal{P}_k$ of paths that begin and end on level $k$, but all its inner vertices are above level $k$ (fig. \ref{fig:1a} and \ref{fig:1b}).
	
	By construction of $c$ every such path $p$ contains at least one string from $\mathcal{S}$ and if it contains more than one string, then in $\sigma$ they go sequentially and in the same order in which $p$ visits them. This allows us to treat such paths as already constructed superstrings of the corresponding subsets of $\mathcal{S}$. From this point of view, we can naturally identify every merge of  some superstrings $s$ and $t$ with $|\overlap(s,t)| = k$ performed by $A$ with situation when for the corresponding paths $p_s, p_t \in \mathcal{P}_k$ there is a path $p \in \mathcal{P}_{k-1}$ containing $p_s \circ p_t$ as subpath (fig. \ref{fig:1c}). It is easy to see, that in general $s$ and $t$ can be merged only if the corresponding paths {\em touch}, i.e. a head vertex of $p_s$ is also a tail vertex of  $p_t$ (of course, this fact doesn't imply that this paths will be merged, as there, for example, may be another path $p_r \in \mathcal{P}_k$ such that $p_r$ and $p_t$ touch and there is $p \in \mathcal{P}_{k-1}$ which contains $p_r \circ p_t$). It is also clear, that if $A$ at some step doesn't merge a pair of superstrings $s, t$ with strictly maximal overlap of the length~$k$, then corresponding paths $p_s, p_t \in \mathcal{P}_k$ touch at some vertex $v$, but there are paths $p'_s, p'_t \in \mathcal{P}_{k-1}$ such that $p'_s$ ends with $p_s \circ (v, \suff(v))$ and $p'_t$ starts with $(\pref(v), v) \circ p_t$. In other words, $p_s$ isn't merged with any other path with start in $v$, $p_t$ isn't merged with any other path with end in $v$ and both are contained in the different paths in $\mathcal{P}_{k-1}$.
	
	In this way, it is sufficient to show that such situations don't happen, to prove that $A$ is greedy. Let us suppose the opposite and consider the mentioned paths $p_s$, $p_t$, $p'_s$, $p'_t$ and the vertex $v$. By definition of $v$ it has income down-arc and outgoing up-arc and also income up-arc and outgoing down-arc, which by construction of $D$ can be only in the case when $v$ is the last chance of the corresponding component $\mathcal{C} \ni v$ to be connected to the rest arcs in $D$. It immediately follows, that all component $\mathcal{C}$ (and hence the paths  $p_s$ and $p_t$) lies in some path $p_{\mathcal{C}} \in \mathcal{P}_{k-1}$, which contradicts the definition of $p'_s$ and $p'_t$. This contradiction completes the proof. $\square$
	
	\begin{figure}[h]
		\centering
		\begin{subfigure}[t]{0.3\textwidth}
			\includegraphics[width=\textwidth]{gha_is_greedy_img/fig1.png}
			\caption{Cycle $c$ for $\mathcal{S}$ containing 2 strings of the length 3 and one string of the length 2}
			\label{fig:1a}
		\end{subfigure}
		\hfil
		\begin{subfigure}[t]{0.3\textwidth}
			\includegraphics[width=\textwidth]{gha_is_greedy_img/fig2.png}
			\caption{Colored arcs is $\mathcal{P}_2$, which contains two paths $p_1$ and $p_2$ touched at $v$ and painted in red and blue, respectively}
			\label{fig:1b}
		\end{subfigure}
		\hfil
		\begin{subfigure}[t]{0.3\textwidth}
			\includegraphics[width=\textwidth]{gha_is_greedy_img/fig3.png}
			\caption{Colored arcs is $\mathcal{P}_1$. We see, that $p_1$ and $p_2$ from previous figure now merged and are contained in new $p_1$. This means that corresponding strings are also merged.}
			\label{fig:1c}
		\end{subfigure}
	\caption{}
	\end{figure}
	
\end{document}