\documentclass[11pt]{article}
\usepackage{fullpage}
\usepackage[utf8]{inputenc}
\usepackage[english]{babel}
\usepackage{amsmath, amsfonts, amssymb, amsthm}

\usepackage{setspace}
\usepackage{relsize}
\usepackage{latexsym}
\usepackage{enumitem}

\usepackage{tikz}
\usetikzlibrary{calc,arrows,shapes,backgrounds,patterns,fit,decorations,decorations.pathmorphing}

\usepackage[colorinlistoftodos]{todonotes}
\usepackage[hidelinks]{hyperref}

\usepackage{algorithm}
\usepackage[noend]{algpseudocode}

\newtheorem{lemma}{Lemma}
\newtheorem{theorem}{Theorem}
\newtheorem{proposition}{Proposition}
\newtheorem{claim}{Claim}
\newtheorem{definition}{Definition}
\newtheorem*{statement}{Statement}
\newtheorem*{remark}{Remark}


\DeclareMathOperator{\overlap}{overlap}
\DeclareMathOperator{\pref}{pref}
\DeclareMathOperator{\suff}{suff}
\DeclareMathOperator{\ein}{in}
\DeclareMathOperator{\eout}{out}
\DeclareMathOperator{\bal}{\Delta}
\DeclareMathOperator{\lvl}{level}
\DeclareMathOperator{\defn}{def}
\DeclareMathOperator{\poly}{poly}
\DeclareMathOperator{\inp}{input}
\DeclareMathOperator{\indegree}{indegree}
\DeclareMathOperator{\outdegree}{outdegree}

\renewcommand{\geq}{\geqslant}
\renewcommand{\leq}{\leqslant}
\newcommand{\sedge}[2]{$({\tt #1}, {\tt #2})$}
\newcommand{\cld}{D_{cl}}
\newcommand{\grd}{D_{gr}}
\newcommand{\cldr}[1]{D_{cl}(#1)}
\newcommand{\grdr}[1]{D_{gr}(#1)}
\newcommand{\clgraph}{G_{cl}}
\newcommand{\grgraph}{G_{gr}}


\tikzstyle{hgedge}=[->,gray!40!white]
\tikzstyle{anypath}=[->,dashed]
\tikzstyle{vertex}=[draw,ellipse,inner sep=0.5mm]
\tikzstyle{inputvertex}=[draw,rectangle,inner sep=1mm]

\newenvironment{mypic}{\begin{center}\begin{tikzpicture}[>=latex,line width=.3mm]}{\end{tikzpicture}\end{center}}


\begin{document}
\listoftodos

\sloppy
\begin{titlepage}
%\date{}
\title{Collapsing Superstring Conjecture}
\author{
Alexander Golovnev\thanks{Harvard University}
\and
Alexander~S.~Kulikov\thanks{Steklov Institute of Mathematics at St.~Petersburg, Russian Academy of Sciences}
\and
Alexander Logunov\footnotemark[2]
\and
Ivan Mihajlin\thanks{University of California, San Diego}
}
\maketitle
\thispagestyle{empty}

\begin{abstract}
In the Shortest Common Superstring (SCS) problem, one is given a collection of strings, and needs to find a shortest string containing each of them as a~substring. SCS admits $2\frac{11}{23}$-approximation in polynomial time [Mucha~?].{\todo{AK: Sasha G., should we give the reference here?}} While this algorithm and its analysis are technically involved, the $30$ years old Greedy Conjecture claims that the trivial and efficient Greedy Algorithm gives a~$2$-approximation for SCS. The Greedy Algorithm repeatedly merges two strings with the largest intersection into one, until only one string remains.

We develop a graph-theoretic framework for studying approximation algorithms for SCS. In this framework, we give a (stronger) counterpart to the Greedy Conjecture: We conjecture that the presented in this paper Greedy Hierarchical Algorithm gives a $2$-approximation for SCS. This algorithm is almost as simple as the standard Greedy Algorithm, and we suggest a~combinatorial approach for proving this conjecture. We support the conjecture by showing that the Greedy Hierarchical Algorithm gives a~$2$-approximation in the case when all input strings have length at most $3$ (which until recently had been the only case where the Greedy Conjecture was proven). We also tested our conjecture on tens of thousands{\todo{AK: Sasha G., let's replace by millions everywhere?}} of instances of SCS.

Except for its conjectured good approximation ratio, the Greedy Hierarchical Algorithm finds \emph{exact} solutions for the special cases where we know polynomial time (not greedy) exact algorithms: (1)~when the input strings form a spectrum of a~string (2)~when all input strings have length at most $2$.

\todo[inline]{AK: update abstract: curious structural property, curious connection to greedy}
\end{abstract}
\end{titlepage}

\tableofcontents
\todo[inline]{AK: remove table of contents before submitting}

\input intro
\input definitions
\input collapsing
\input greedy_hierarchical
\input conclusion
\bibliographystyle{alpha}
\bibliography{main}
\appendix
\input greedy_special_case
\input collapsing_special_case
\end{document}