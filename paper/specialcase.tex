In this subsection, we support the Collapsing Superstring Conjecture by proving it in a~special case when all input strings have length~3.

\subsection{Main idea}

The proof, with the exception of the last (most complex) part, will also be true for the general superstring problem.

Define $V(s)$ as the set of vertices examined by the greedy algorithm strictly before the examination of vertex $s$ (including $s$ itself). This definition will be the same for the algorithm \ref{alg:collapser}, because they both examine vertices in descending order of levels and in the same order at the same levels.

Both algorithms work with the set named $D$. We denote the set $D$ from the collapsing algorithm as $\cld{}$ and the set $D$ from the greedy algorithm as $\grd{}$. We also introduce the notation $\clgraph = (V, \cld), \grgraph = (V, \grd)$. Finally, for collapsing algorithm we define the set $\cldr{s}$, consisting of edges of $D$ which are incident to at least one vertex from $ V (s) \backslash \{s \} $, where $D$ is considered as a set of edges strictly before the examination of vertex $s$. We define $\grdr{s}$ similarly.

\begin{remark}
Generally speaking, $\grdr{s}=\grd$, so for the greedy algorithm this designation is introduced only for uniformity. On the other hand, $\cld$ is not necessarily equal to $\cldr{s}$, because $\clgraph{}$ initially consists of edges connecting all vertices from $ {\cal S} $ to $ \varepsilon $.
\end{remark}

Let's formulate the main assertion.

\begin{statement}
    $\forall s \in V$ $\cldr{s}=\grdr{s}$ holds.
\end{statement}

Define the first vertex in $G$ which is examined by both algorithms as $s_0$. Note that for $s_0$ the formulated statement is obvious, because $\cldr{s_0} = \grdr{s_0} = \varnothing $. Also note that for $\varepsilon $ this statement is equivalent to the Collapsing Superstring Conjecture, since $\cldr{\varepsilon}=\cld, \grdr{\varepsilon}=\grd $ by virtue of $ V (\varepsilon) = V $.

If $s$ and $t$ are the neighboring vertices in the order of their consideration by both algorithms, then
\begin{equation}
\label{eqn:stepcase}
    D(t) = D(s) \sqcup \bigsqcup_{i=1}^{a} (\pref(s) \to s) \sqcup \bigsqcup_{j=1}^{b} (s \to \suff(s))
\end{equation}

for some non-negative $ a $ and $ b $. It means that after performing each iteration sets $ D (s) $ change only slightly.

Therefore, we are going to prove this assertion by induction on vertices in the order of their consideration by both algorithms. The base case of induction has already been proved. It remains only to prove the step case.

\subsection{Proof of the step case (simple part)}

According to (\ref{eqn:stepcase}), it is sufficient to prove the following theorem:

\begin{theorem}
    For each vertex $ s $ of a level at least $ 1 $, if the collapsing algorithm after its examination leaves $ a_{cl} $ edges $ (\pref (s) \to s) $ and $ b_{cl} $ edges $ (s \to \suff(s) ) $, and the greedy algorithm leaves $ a_{gr} $ and $ b_{gr} $ edges, respectively, then $ a_{cl} = a_{gr}, b_{cl} = b_{gr} $.
\end{theorem}

\begin{proof}

We denote $ a_{cl} $ as $ a $, and $ b_{cl} $ as $ b $ for short. 
    
First, we will find out what values can take $ a $ and $ b $ for each vertex from the formulated theorem.
\begin{lemma}
\label{lem:edgecases}
The cases $a \ge 2, b \ge 1$ and $a \ge 1, b \ge 2$ are not realized.
\end{lemma}

\begin{proof}
In both cases, the collapsing algorithm has the ability to make at least one collapse from the vertex $ s $. We will show that this collapse is correct, that is, $ a $ and $ b $ must be reduced by at least one. For this it suffices to prove that $ s $, $ \pref (s) $ and $ \suff (s) $ after the collapse remain weakly connected.

Let the level of $ s $ be at least $ 2 $. Then, after the collapse vertices $ \pref(s) $ and $ \suff (s) $     still will be weakly connected through the vertex $ \pref (\suff (s)) = \suff (\pref (s)) $. Since either $ a \ge 2 $ or $ b \ge 2 $, the vertex $ s $ will remain weakly connected with $ \pref (s) $ or $ \suff (s) $, and hence with both of these vertices (see the example in the figure \ref{fig:collapsea2b1}).

\begin{figure}[ht]
\begin{center}

\begin{tikzpicture}[xscale=1.3] 
  \begin{scope}
    \foreach \i in {0,...,1} {
        \node[inputvertex] (s\i) at (\i*5+1,3) {$s$};
        \node[inputvertex] (ps\i) at (\i*5+0,2) {$\pref(s)$};
        \node[inputvertex] (ss\i) at (\i*5+2,2) {$\suff(s)$};
        \node[inputvertex] (t\i) at (\i*5+1,1) {$t$};
    }
    \draw [-] (3,2) -- (3.06,2);
    \draw [->,decorate,decoration={snake, post length=0.4mm}] (3.06,2) -- (4,2);
    
    \path (ps0) edge[bend left=15, ->] (s0);
    \path (ps0) edge[bend left=45, ->] (s0);
    \path (s0) edge[->] (ss0);
    
    \path (ps1) edge[bend left=15, ->, dashed] (s1);
    \path (ps1) edge[bend left=45, ->] (s1);
    \path (s1) edge[->, dashed] (ss1);
    \path (ps1) edge[->] (t1);
    \path (t1) edge[->] (ss1);
  \end{scope}
\end{tikzpicture}
\end{center}
    
\caption{Example, in which $a \ge 2, b \ge 1$. The string $t$ denotes $\pref(\suff(s))$.}\label{fig:collapsea2b1}
\end{figure}
There remains the case when the level $s$ equals $1$. But then $ \pref (s) = \suff (s) = \varepsilon $ and, obviously, $s$ and $ \varepsilon $ still will be connected after the collapse.
\end{proof}

From the lemma \ref{lem:edgecases} it follows that there remains only four cases: $ a> 0, b = 0 $; $ a = 0, b> 0 $; $ a = 0, b = 0 $ and $ a = 1, b = 1 $.

By the induction hypotesis, $ D_{cl} (s) = D_{gr} (s) $. This means, in particular, that the sets of edges in which $ s $ is the lowest vertex coincide in both algorithms, because all vertices with strictly higher levels have already been examined. Since the balance of the vertex $s$ in both algorithms must eventually become equal to $ 0 $, the edges in which $ s $ is the top vertex must contribute the same to the $ s $ balance in both algorithms. In other words,

\begin{equation}
\label{eqn:balance}
    a-b = a_{gr}-b_{gr}.
\end{equation}

Therefore, it suffices to prove only one of the equalities $ a = a_{gr} $ or $ b = b_{gr} $.

\textbf {Cases 1 and 2.} Note that in the notation of the greedy algorithm, before examination of the vertex $ s $, we have $ \bal (s) \ne 0 $, because $ a_{gr} - b_{gr} = a-b \ne 0 $. Hence, the greedy algorithm either draws $ \bal (s) $ edges from $ s $ if $ \bal (s)> 0 $ (case 1), or draws $ - \bal (s) $ edges down from $ s $ if $ \bal (s) <0 $ (case 2). Either way, in the first case $ b_{gr} = 0 $, and in the second case $ a_{gr} = 0 $, as required.

The first two cases are done. In the remaining two cases $ \bal (s) = 0 $. This means that the greedy algorithm, during examination of the vertex $ s $ either draws two edges $ (\pref (s) \to s) $ and $ (s \to \suff (s)) $ from $ s $, or does nothing.

\textbf {Case 3.} We need to show that the greedy algorithm will not do anything when examining the vertex $ s $. Suppose, on the contrary, that it has drawn two edges from $ s $. This happens if and only if all other vertices in the weakly connected component of $ s $ are already examined. In other words, if we denote the weakly connected component of $ s $ in $ G_{gr} $ for $ C_{gr} (s) $, then $ C_{gr} (s) \subset V (s) $.

Now we denote the weakly connected component of $ s $ in $ G_{cl} $ as $ C_{cl} (s) $ and show that $ C_{cl} (s) = C_{gr} (s) $. Since $ \varepsilon \notin C_{gr} (s) $, this means that in the course of the collapsing algorithm, a connected component that does not contain $ \varepsilon $ appears in the graph. But this is impossible, and this is the contradiction.

Let $ v \in C_{gr} (s) $. Hence, in $ G_{gr} $ there exists a path from $ s $ to $ v $ which contains only edges from $ D_{gr} (s) $, because there are no other edges in the graph. By the induction hypothesis, $ D_{cl} (s) = D_{gr} (s) $. Hence, in $ G_{cl} $ there is also a path from $ s $ to $ v $, and therefore $ v \in C_{cl} (s) $.

Let $ v \in C_{cl} (s) $. Hence, in $ G_{cl} $ there is a path from $ s $ to $ v $. If it contains only edges from $ D_{cl} (s) = D_{gr} (s) $, then it connects $ s $ with $ v $ and $ G_{gr} $, and then $ v \in C_{gr} (s) $. Suppose that there exists an edge in this path not from $ D_{cl} (s) $. Let's take the first such edge. Let it go from the vertex $ t $ to the vertex $ u $.

Hence both vertices $ t $ and $ u $ do not lie in $ V (s) \backslash \{s \} $. We also note that $ t \ne s $. Indeed, otherwise, either $ u = \suff (s) $ or $ u = s + c $ for some symbol $ c $. But $ b = 0 $ forbids the first case, and in the second case $ (s \to s + c) \ in D_{cl} (s) $. Therefore, $ t \notin V (s) $.

Since $ (t \to u) $ is the first edge on the path that does not lie in $ D_{cl} (s) $, all the edges in the path before it lie in $ D_{cl} (s) $. Let $ (r \to t) $ be the last such edge. Since $ D_{cl} (s) = D_{gr} (s) $, the entire path from $ s $ to $ t $ lies in $ D_{gr} (s) $, therefore $ t \in C_{gr} (s) $. But we just got that $ t \notin V (s) $, and it is a contradiction with $ C_{gr} (s) \subset V (s) $.

Hence $ C_{cl} (s) = C_{gr} (s) $. This leads to the desired contradiction.

\textbf {Case 4.} It is necessary to show that if the collapsing algorithm leaves two edges $ (\pref (s) \to s) $ and $ (s \to \suff (s)) $ after examination of $ s $, then greedy algorithm during examination of $ s $ draws these two edges from it. As already noted, this happens if and only if all the other vertices in the weakly connected component $ s $ have already been examined. So, this time we need to prove that $ C_{gr} (s) \subset V (s) $.

Let it not be so, and there is $ t \in C_{gr} (s) \backslash V (s) $.

Now we denote by $ C'_{cl} (s) $ the weakly connected component of $s$ in the graph $ G'_{cl} = (V, D_{cl} \backslash \{(\pref (s) \to s), (s \to \suff (s)) \}) $. Similarly to the previous case, it turns out that $ C_{gr} (s) \ subset C'_{cl} (s) $. Hence,

\begin{equation}
\label{eqn:contradiction}
    \exists t \in C'_{cl}(s) \backslash V(s).
\end{equation}
We will refute the statement \ref{eqn:contradiction}, and we will do this only for the $3$-SCS instance.

We denote the level of the vertex $ s $ as $ \lvl (s) $ and consider all possible levels of the vertex $ s $.

\subsection{Analysis of vertices of the level \texorpdfstring{$3$}{3}}

In this case $ s \in {\cal S} $, and vertex $ s $, is incident only to edges $ (\pref (s) \to s)$ and $(s \to \suff (s)) $. The collapse leaves only one instance of each such edge. Hence, in the graph $ G'_{cl} $ $ C'_{cl} (s) = \{s \} $. This contradicts (\ref{eqn:contradiction}).

\subsection{Analysis of vertices of the level \texorpdfstring{$2$}{2}}

Let $ t \in C'_{cl} (s) $. Denote by $ \operatorname{dist} (t) $ the length of the shortest path from $ s $ to $ t $ in $ G'_{cl} $. We take a vertex with minimal $ \operatorname{dist} (t) $ among all vertices in the set $ C'_{cl} (s) \backslash V (s) $. Consider the path $ P = v_1 \to v_2 \to \ldots \to v_n $, which connects $ s $ and $ t $.

Note that if $ \lvl (u) = 1 $, then $ u \notin V (s) $. Therefore $ \lvl (v_i) \ge 2 $ holds for $ i<n $, otherwise there would be a vertex with smaller $ \operatorname {dist} $. Since $ | \lvl (r) - \lvl (u) | = 1 $ is fulfilled for any edge $ (r \to u) $, at the beginning of the path the levels of vertices alternate: $ \lvl (v_1) = 2 $, $ \lvl (v_2) = 3 $, $ \lvl (v_3) = 2 $, and so on (see the example in the figure \ref{fig:lvl2path6}).

\begin{figure}[ht]
\begin{center}

\begin{tikzpicture}[xscale=1.3] 
  \begin{scope}
    \foreach \x/\y/\n in {1/2/v1, 2/3/v2, 3/2/v3, 4/3/v4, 5/2/v5, 6/1/v6}
      \node[inputvertex] (\n) at (\x,\y) {\tt \n};
    
    \node[vertex] (e) at (3,0) {$\varepsilon$};
    
    \node[draw=none] at (0,0) {};
    
    \foreach \y in {0,...,3}
      \node[draw=none] at (-1,\y) {\y};
      
    \path (v1) edge[->] (v2);
    \path (v2) edge[->] (v3);
    \path (v3) edge[->] (v4);
    \path (v4) edge[->] (v5);
    \path (v5) edge[->] (v6);
  \end{scope}
\end{tikzpicture}
\end{center}

\caption{Example of the path $P$ for $n=6$.}\label{fig:lvl2path6}
\end{figure}

We also note that if $ \lvl (v_{2k}) = 3 $, then the pair of edges $ v_{2k-1} \to v_{2k} \to v_{2k + 1} $ will be <<reflected>> by the collapse from $ v_{2k} $ to some vertex $ u_{2k} $. In addition, in the graph $ G_{cl} $, there are edges $ (u_0 \to v_1) $ and $ (v_1 \to u_2) $, where $ u_0 = \pref (s) $ and $ u_2 = \suff (s) $. We add to this that if there is a pair of edges $ (u_{2k} \to v_{2k + 1}) $ and $ (v_{2k + 1} \to u_{2k + 2}) $ in the graph, then there is a vertex $ u_{2k + 1} $, into which these edges can be reflected by another collapse.

In view of the fact that the edges $ (v_{2k-1} \to v_{2k}) $ and $ (v_{2k} \to v_{2k + 1}) $ are the only ones that are incident to the vertex $ v_{2k} \in S $, two instances of each of them existed in the graph from the very beginning. Moreover, since $ \lvl (s) = 2 $, all the vertices of the level $ 3 $ have already been examined, and therefore at least one collapse has been performed from each vertex $ v_ {2k} $ of the level $ 3 $. Thus, in the graph at some time there existed edges $ (v_{2k-1} \to u_{2k}) $ and $ (u_{2k} \to v_{2k + 1}) $ (see the example in the figure \ref {fig:lvl2pathuv}).

\begin{figure}[ht]
\begin{center}

\begin{tikzpicture}[xscale=1.3] 
  \begin{scope}
    \foreach \x/\y/\n in {1/2/v1, 2/3/v2, 3/2/v3, 4/3/v4, 5/2/v5, 6/1/v6}
      \node[inputvertex] (\n) at (\x,\y) {\tt \n};
    \foreach \x/\y/\n in {0/1/u0, 1/0/u1, 2/1/u2, 3/0/u3, 4/1/u4, 5/0/u5}
      \node[inputvertex] (\n) at (\x,\y) {\tt \n};
    
    \foreach \y in {0,...,3}
      \node[draw=none] at (-1,\y) {\y};
      
    \path (v1) edge[->] (v2);
    \path (v2) edge[->] (v3);
    \path (v3) edge[->] (v4);
    \path (v4) edge[->] (v5);
    \path (v5) edge[->] (v6);
    
    \draw[->] (u0) to node [sloped] {\(\not\quad\)} (v1);
    \draw[->] (v1) to node [sloped] {\(\not\quad\)} (u2);
    \path (u2) edge[->, dashed] (v3);
    \path (v3) edge[->, dashed] (u4);
    \path (u4) edge[->, dashed] (v5);
  \end{scope}
\end{tikzpicture}
\end{center}

\caption{The mutual arrangement of vertices $ v_i $ and $ u_i $ for $ n = 6 $. Note that $u_1=u_3=u_5=\varepsilon$.}\label{fig:lvl2pathuv}
\end{figure}

Now we will exploit the fact that the pair of edges $ (u_0 \to v_1) $ and $ (v_1 \to u_2) $ has not been collapsed. This means that after this collapse weak connectivity would be broken.

Note that $ u_1 = \varepsilon $; in addition, after the collapse $ u_0 $ and $ u_2 $ would be weakly connected through $ \varepsilon $. This means, in particular, that after the collapse $ v_1 $ can't be connected with $ u_2 $, otherwise the mutual weak connectivity would not be violated for any of the pairs of vertices.

\begin{lemma}
\label{lem:pathcollapse}
For all $ k $ such that $ 3 \le 2k + 1 \le n $ at least one collapse was performed from the vertex $ v_{2k + 1} $.\end{lemma}
\begin{proof}
We will prove this by induction on $ k $. $ k = 0 $ will be the the induction base.

Suppose that at least one collapse is made from the vertex $ v_{2k '+ 1} $ for all $ k' $ such that $ 3 \le 2k '+ 1 <2k + 1 $. Suppose that not a single collapse was made from the vertex $ v_{2k + 1} $. This means that the edge $ (u_{2k} \to v_{2k + 1}) $ is present in the graph at the current iteration.

By definition, after a correct collapse from the vertex $ v_{2k '+ 1} $, a pair of edges $ (u_{2k'} \to u_{2k '+ 1}) $ and $ (u_{2k' + 1} \to u_{2k '+ 2}) $. Moreover, any of these edges can disappear from the graph only after the collapse from the vertex $ u_ {2k '+ 1} $. But the level of this vertex is $ 1 $, and therefore it has not yet been visited. Therefore, both edges remain in the graph at the current iteration.

Then there exists a weak path $ Q $ in the graph of the following form:

$$
Q = v_1 - v_2 - \ldots - v_{2k+1} - u_{2k} - u_{2k-1} - \ldots u_2.
$$

It means that $v_1$ and $u_2$ were weakly connected from the very beginning. Contradiction.
\end{proof}

From the lemma \ref{lem:pathcollapse} it follows that on the current iteration in the graph there must exist a weak path $ u_2 - \ldots - u_{2k} - u_{2k + 1} $ for any $ k $ such that $ 3 \le 2k + 1 \le n $.

Now we note that either $ \lvl (v_{n}) = 2 $, or $ \lvl (v_{n}) = 1 $. Let us consider both cases.

If $ \lvl (v_{n}) = 1 $, then, since from every vertex $ v_{2k + 1} $ there occurs at least one collapse, there exists a pair of edges $ u_{2k} \to u_{2k + 1} ) $ and $ (u_{2k + 1} \to v_n) $ in the graph. Hence, there exists a weak path of the form $ v_1 - v_2 - \ldots - v_n - u_{2k + 1} - \ldots - u_3 - u_2 $, and the vertices $ v_1 $ and $ u_2 $ turn out to be weakly connected. Contradiction (see the example in the figure \ref {fig:lvl2n6pathvu}).

If $ \lvl (v_{n}) = 2 $, then $ n = 2k + 1 $ for some $ k $. Hence, by the lemma \ref{lem:pathcollapse}, at least one collapse was made from the vertex $ v_{2k + 1} $. On the other hand, the vertex $ v_{2k + 1} \notin V (s) $, and hence it has not been examined yet. Contradiction (see the example in the figure \ref{fig:lvl2n5pathvu}).

\begin{figure}[ht]
\begin{center}

\begin{tikzpicture}[xscale=1.3] 
  \begin{scope}
    \foreach \x/\y/\n in {1/2/v1, 2/3/v2, 3/2/v3, 4/3/v4, 5/2/v5, 6/1/v6}
      \node[inputvertex] (\n) at (\x,\y) {\tt \n};
    \foreach \x/\y/\n in {0/1/u0, 1/0/u1, 2/1/u2, 3/0/u3, 4/1/u4, 5/0/u5}
      \node[inputvertex] (\n) at (\x,\y) {\tt \n};
    
    \foreach \y in {0,...,3}
      \node[draw=none] at (-1,\y) {\y};
      
    \path (v1) edge[->, ultra thick] (v2);
    \path (v2) edge[->, ultra thick] (v3);
    \path (v3) edge[->, ultra thick] (v4);
    \path (v4) edge[->, ultra thick] (v5);
    \path (v5) edge[->, ultra thick] (v6);
    
    \draw[->] (u0) to node [sloped] {\(\not\quad\)} (v1);
    \draw[->] (v1) to node [sloped] {\(\not\quad\)} (u2);
    \path (u2) edge[->, ultra thick] (u3);
    \path (u3) edge[->, ultra thick] (u4);
    \path (u4) edge[->, ultra thick] (u5);
    \path (u5) edge[->, ultra thick] (v6);
  \end{scope}
\end{tikzpicture}
\end{center}

\caption{Weak path which connects $v_1$ and $u_2$ for $n=6$.}\label{fig:lvl2n6pathvu}
\end{figure}

\begin{figure}[ht]
\begin{center}

\begin{tikzpicture}[xscale=1.3] 
  \begin{scope}
    \foreach \x/\y/\n in {1/2/v1, 2/3/v2, 3/2/v3, 4/3/v4, 5/2/v5}
      \node[inputvertex] (\n) at (\x,\y) {\tt \n};
    \foreach \x/\y/\n in {0/1/u0, 1/0/u1, 2/1/u2, 3/0/u3, 4/1/u4}
      \node[inputvertex] (\n) at (\x,\y) {\tt \n};
    
    \foreach \y in {0,...,3}
      \node[draw=none] at (-1,\y) {\y};
      
    \path (v1) edge[->, ultra thick] (v2);
    \path (v2) edge[->, ultra thick] (v3);
    \path (v3) edge[->, ultra thick] (v4);
    \path (v4) edge[->, ultra thick] (v5);
    
    \draw[->] (u0) to node [sloped] {\(\not\quad\)} (v1);
    \draw[->] (v1) to node [sloped] {\(\not\quad\)} (u2);
    \path (u2) edge[->, ultra thick] (u3);
    \path (u3) edge[->, ultra thick] (u4);
    \path (u4) edge[->, ultra thick] (v5);
  \end{scope}
\end{tikzpicture}
\end{center}

\caption{Weak path which connects $v_1$ and $u_2$ for $n=5$.}\label{fig:lvl2n5pathvu}
\end{figure}

\subsection{Analysis of vertices of the level \texorpdfstring{$1$}{1}}

In this case, all the vertices at the levels $ 2 $ and $ 3 $ are already considered.

Note that the collapse from the vertex $ s $ can potentially violate only the connectivity of $ s $ and $ \varepsilon $. Therefore $ \varepsilon \notin C'_{cl} (s) $. Let's take an arbitrary vertex $ t $ from the statement \ref{eqn:contradiction}. From what has just been said, it follows that $ \lvl (t) = 1 $.

In view of the fact that $ t $ is connected with $ s $, the vertex $ t $ is incident to at least one edge. But it can not be connected with the vertex on the level $ 0 $, because only $ \varepsilon $ is located at this level, and $ \varepsilon \notin C'_{cl} (s) $. Hence, $ t $ is incident to at least one edge leading to the level $ 2 $. Moreover, since $ \bal (t) = 0 $, at least one such edge comes to $ t $, and at least one such edge leaves $ t $.

Starting from this point, we will call all vertices as strings which they correspond to. So, $ t $ corresponds to some string $ \tt b $ consisting of one symbol. We have just shown that there are edges $ ({\tt xb} \to {\tt b}) $ and $ ({\tt b} \to {\tt bc}) $ for some symbols $ {\tt x} $ and $ {\tt c} $.

First we prove several auxiliary lemmas.

\begin{lemma}
\label{lem:nobeps}
In the graph there neither was an edge $ ({\tt b} \to \varepsilon) $ nor an edge $ (\varepsilon \to {\tt b}) $.
\end{lemma}
\begin{proof}
Let at least one of such edges appear in the graph at some iteration. According to the algorithm, such an edge can disappear from the graph only after the collapse in the vertex $ {\tt b} $. Since this vertex has not yet been examined, such an edge is present in the graph at this moment. But this means that $ \varepsilon \in C'_{cl} (s) $. Contradiction.
\end{proof}

\begin{lemma}
\label{lem:bccsave}
Each edge of the form \sedge{bc}{c}, \sedge{b}{bc}, or \sedge{ab}{b} that ever appeared in the graph was not collapsed until the current iteration.
\end{lemma}
\begin{proof}
Each edge of the form \sedge{bc}{c} can only be collapsed with the edge \sedge{b}{bc}. Then, if a collapse actually occurs from the vertex $ {\tt bc} $, it leads to the appearance of the edge $ ({\tt b} \to \varepsilon) $. But according to the lemma \ref{lem:nobeps}, there never were any such edges in the graph. Contradiction.

The proofs for \sedge{b}{bc} and \sedge{ab}{b} are the same.
\end{proof}

\begin{lemma}
\label{lem:bbcone}
At the current iteration, there is exactly one instance of the edge \sedge{b}{bc}.
\end{lemma}
\begin{proof}
Let there be at least two such edges. Then at least two different edges must come from $ {\tt bc} $.

Suppose that there exists at least one edge \sedge{bc}{c}. But then, if we make a collapse from the vertex $ {\tt bc} $, the vertices $ {\tt b} $, $ {\tt c} $ and $ {\tt bc} $ still remain weakly connected by edges \sedge{b}{bc}, $ ({\tt b} \to \varepsilon) $ and $ (\varepsilon \to {\tt c}) $. Hence, this collapse is correct (see figure \ref{fig:lvl1lemmabcc}). But then the collapsing algorithm during examination of the vertex $ {\tt bc} $, should have made at least one more collapse. It contradicts the fact that it does the maximum possible number of correct collapses.

\begin{figure}[ht]
\begin{center}

\begin{tikzpicture}[xscale=1.2] 
  \begin{scope}
    \foreach \i in {0,...,1} {
        \foreach \x/\y/\n in {1/1/b, 2/2/bc, 3/1/c}
            \node[vertex] (\n\i) at (\i*4+\x,\y) {\tt \n};
        \node[vertex] (e\i) at (\i*4+2,0) {$\varepsilon$};
    }
    
    \foreach \y in {0,...,2}
      \node[draw=none] at (0,\y) {\y};
    
    \draw [-] (3.5,1.5) -- (3.56,1.5);
    \draw [->,decorate,decoration={snake, post length=0.4mm}] (3.56,1) -- (4.5,1);
    
    \path (b0) edge[->, bend left=15] (bc0);
    \path (b0) edge[->, bend right=15] (bc0);
    \path (bc0) edge[->] (c0);

    \path (b1) edge[->, bend left=15] (bc1);
    \path (b1) edge[->, dashed, bend right=15] (bc1);
    \path (bc1) edge[->, dashed] (c1);
    \path (b1) edge[->] (e1);
    \path (e1) edge[->] (c1);
  \end{scope}
\end{tikzpicture}
\end{center}

\caption{Correctness of the collapse in case when the edge \sedge{bc}{c} is present.}\label{fig:lvl1lemmabcc}
\end{figure}

Hence, there is at least one edge \sedge{bc}{bcd}. Then $ {\tt bcd} \in {\cal S} $, and then initially there were two pairs of edges \sedge{bc}{bcd} and \sedge{bcd}{cd} in the graph. So, from {\tt bcd}, at least one collapse was made, and it led to the appearance of the edge \sedge{bc}{c}. But at the current iteration the edge \sedge{bc}{c} can't exist, so it was also collapsed. It's a contradiction with the lemma \ref{lem:bccsave}.
\end{proof}

\begin{figure}[ht]
\begin{center}

\begin{tikzpicture}[xscale=1.2] 
  \begin{scope}
    \foreach \i in {0,...,2} {
        \foreach \x/\y/\n in {1/1/b, 2/2/bc, 3/1/c, 3/3/bcd}
            \node[vertex] (\n\i) at (\i*4+\x,\y) {\tt \n};
        \node[vertex] (e\i) at (\i*4+2,0) {$\varepsilon$};
    }
    
    \foreach \y in {0,...,3}
      \node[draw=none] at (0,\y) {\y};
    
    \draw [-] (3.5,1.5) -- (3.56,1.5);
    \draw [->,decorate,decoration={snake, post length=0.4mm}] (3.56,1.5) -- (4.5,1.5);
    
    \draw [-] (7.5,1.5) -- (7.56,1.5);
    \draw [->,decorate,decoration={snake, post length=0.4mm}] (7.56,1.5) -- (8.5,1.5);
    
    \path (b0) edge[->, bend left=15] (bc0);
    \path (b0) edge[->, bend right=15] (bc0);
    \draw[->] (bc0) to node [sloped] {\(\not\quad\)} (c0);
    
    \path (b1) edge[->, bend left=15] (bc1);
    \path (b1) edge[->, bend right=15] (bc1);
    \draw[->] (bc1) to node [sloped] {\(\not\quad\)} (c1);
    \path (bc1) edge[->, bend left=15] (bcd1);
    \path (bc1) edge[->, bend right=15] (bcd1);
    
    \path (b2) edge[->, bend left=15] (bc2);
    \path (b2) edge[->, bend right=15] (bc2);
    \path (bc2) edge[->, ultra thick] (c2);
    \path (bc2) edge[->, bend left=15] (bcd2);
    \path (bc2) edge[->, bend right=15, dashed] (bcd2);
  \end{scope}
\end{tikzpicture}
\end{center}

\caption{The scheme of proof of the lemma \ref{lem:bbcone}.}\label{fig:lvl1lemmabbcone}
\end{figure}

So, there is only one edge \sedge{b}{bc}.

\begin{lemma}
\label{lem:bcup}
The vertex $ {\tt bc} $ is weakly connected with at least one vertex at the level $ 3 $.
\end{lemma}
\begin{proof}
Let it not be so. Then $ {\tt bc} $ can contain only edges \sedge{b}{bc}, and only edges \sedge{bc}{c} can come from it. By the lemma \ref{lem:bbcone}, there is only one instance of the edge \sedge{b}{bc}. Hence, there is exactly one instance of the edge \sedge{bc}{c}. But then the algorithm had to collapse this pair of edges when examining $ {\tt bc} $, because this operation would only separate $ {\tt bc} $ from all other vertices without violating the connectivity of $ \varepsilon $ and $ {\cal S} $. Contradiction.
\end{proof}

By lemma \ref{lem:bcup} either there exists $ {\tt d} $ such that there is an edge \sedge{bc}{bcd}, or there is a symbol $ {\tt a} $ such that there is an edge \sedge{abc}{bc}. Suppose there is no such $ {\tt a} $, which means that there exists some $ {\tt d} $.

Then $ {\tt bcd} \in {\cal S} $, and therefore initially there were two instances of edges \sedge{bc}{bcd} and \sedge{bcd}{cd} in the graph. So, at least one collapse was made from {\tt bcd}, and it led to the appearance of the edge \sedge{bc}{c}. The edge \sedge{bc}{c} could not be collapsed by the lemma \ref{lem:bccsave}. Hence, some other edge must come to $ {\tt bc} $, and by lemma \ref{lem:bbcone} it can not be another instance of \sedge{b}{bc}.

Hence, there is an $ {\tt a} $ such that there is an edge \sedge{abc}{bc}. Then $ {\tt abc} \in {\cal S} $, and from $ {\tt abc} $ exactly one collapse was made, which led to the appearance of two edges \sedge{ab}{b} and \sedge{b}{bc}, which were not collapsed at the current moment by the lemma \ref{lem:bccsave}.

Therefore, $ {\tt bc} $ contains at least two edges: \sedge{abc}{bc} and \sedge{b}{bc}. Hence, at least two edges come from $ {\tt bc} $.

\begin{lemma}
\label{lem:bccexist}
At the current iteration, there is at least one edge \sedge{bc}{c}.
\end{lemma}
\begin{proof}
Let it not be so. Then, since at least one edge must come from $ {\tt bc} $, it is an edge of the form \sedge{bc}{bcd}. So, $ {\tt bcd} \in {\cal S} $, initially there were two instances of edges \sedge{bc}{bcd} and \sedge{bcd}{cd} in the graph, and there was made at least one collapse from {\tt bcd}, which led to the appearance of the edge \sedge{bc}{c}. Contradiction.
\end{proof}

So, by the lemma \ref{lem:bbcone}, there is one edge \sedge{b}{bc} in the graph, and by the \ref{lem:bccexist} lemma there is at least one edge \sedge{bc}{c}.

But we assert that this pair of edges should have been collapsed, which will contradict the lemma \ref{lem:bccsave} and the definition of the algorithm which makes the maximum possible number of collapses.

The point is that even after the collapse there will be weak paths $ {\tt b} - \varepsilon - {\tt c} $ and $ {\tt b} - {\tt ab} - {\tt abc} - {\tt bc} $, and hence the connectivity of the vertices $ {\tt b} $, $ {\tt c} $ and $ {\tt bc} $ will be preserved after the collapse (see figure \ref{fig:lvl1final}).
\begin{figure}[ht]
\begin{center}

\begin{tikzpicture}[xscale=1.3] 
  \begin{scope}
    \foreach \x/\y/\n in {1/2/ab, 2/1/b, 2/3/abc, 3/2/bc, 4/1/c}
      \node[vertex] (\n) at (\x,\y) {\tt \n};

    \node[vertex] (e) at (3,0) {$\varepsilon$};
    
    \foreach \y in {0,...,3}
      \node[draw=none] at (0,\y) {\y};
    
    \foreach \s/\t in {ab/b, ab/abc, abc/bc}
      \path (\s) edge[->] (\t);
    
    \draw[->] (b) to node [sloped] {\(\not\quad\)} (bc);
    \draw[->] (bc) to node [sloped] {\(\not\quad\)} (c);
    
    \path (b) edge[->, dashed] (e);
    \path (e) edge[->, dashed] (c);
  \end{scope}
\end{tikzpicture}
\end{center}

\caption{The final state of the graph.}\label{fig:lvl1final}
\end{figure}

In this case, one more correct collapse can be made from $ {\tt bc} $. Contradiction.

This completes the proof of the theorem for the $3$-SCS instance.
\end{proof}
