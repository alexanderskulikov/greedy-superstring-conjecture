\documentclass[14pt,xcolor=dvipsnames]{beamer}
\usetheme{metropolis}
\setsansfont{Ubuntu}
\metroset{progressbar=none,numbering=fraction,sectionpage=progressbar,block=fill}
\setbeamercolor{background canvas}{bg=white}
\usepackage{graphicx}
\graphicspath{{images/}}
\usepackage{booktabs}
\usetikzlibrary{calc,arrows,shapes,backgrounds,patterns,fit,decorations,decorations.pathmorphing}


%\includeonlyframes{tight}

\DeclareMathOperator{\overlap}{overlap}

\colorlet{mc}{mLightBrown}


\tikzstyle{hgedge}=[->,line width=.3mm,gray!50]
\tikzstyle{highhgedge}=[hgedge,mc,line width=2.5pt]
\tikzstyle{anypath}=[->,dashed]
\tikzstyle{vertex}=[draw,rectangle,inner sep=0.5mm]
\tikzstyle{inputvertex}=[draw,rectangle,inner sep=0.5mm,fill=gray!40]


%%%%%%


\title{Collapsing Superstring Conjecture}
\date{}%February 15, 2019}
\author{Alexander Golovnev (Harvard)\\ Alexander S. Kulikov (Steklov Math Inst)\\ Alexander Logunov (St. Petersburg U)\\ Ivan Mihajlin (UCSD)\\ Maksim Nikolaev (St. Petersburg U)\\}
%\institute{Steklov Mathematical Institute at St. Petersburg\\and\\
%St. Petersburg State University\\}

\newenvironment{mypic}{\begin{center}\begin{tikzpicture}[line width=1.5pt,>=latex,scale=1.0]}
{\end{tikzpicture}\end{center}}

\newenvironment{mynewpic}{\begin{center}\begin{tikzpicture}[line width=1.5pt,>=latex]}
{\end{tikzpicture}\end{center}}

\newcommand{\myproblem}[3]{
  \begin{block}{#1}
  \begin{itemize}
  \item[] Input: \hspace{0mm} #2
  \item[] Output: #3
  \end{itemize}
  \end{block}
}

%%%%%%%%%%%%%%%%%%%%%%%%%%%%%%%%%%
%%%%%%%%%%%%%%%%%%%%%%%%%%%%%%%%%%

\begin{document}
\maketitle

\begin{frame}[label=in]{Overview by an Anonymous Reviewer}
\emph{This conjecture seems shockingly strong, since it claims that a certain natural transformation has a
specific solution as its fixed point. Intuitively, it seems to be the kind of conjecture that is either false or relatively easy to prove. Quite surprisingly, the authors' experiments \alert{failed} to produce a counterexample and the conjecture actually holds for some simple cases, but the authors \alert{also failed} to produce a proof for the general case.}
\end{frame}

\input overview
\input hierarchical
\input collapsing
\input hierarchical_conjecture
\input relation

\begin{frame}{Thank you!}
\vfill
\centerline{\alert{\bf \large Thank you for your attention!}}
\vfill
\begin{itemize}
\item \alert{Full text:} \url{arxiv.org/abs/1809.08669}
\item \alert{Visualization:} \url{compsciclub.ru/scs}
\item \alert{Code:} \url{github.com/alexanderskulikov/greedy-superstring-conjecture}
\end{itemize}
\vfill
\end{frame}


\end{document}

